% !TeX root = ../il-screen.tex

\chapter{About this Book}

This book is an introduction to metalogic, aimed especially at
students of computer science and philosophy. ``Metalogic'' is
so-called because it is the discipline that studies logic itself.
Logic proper is concerned with canons of valid inference, and its
symbolic or formal version presents these canons using formal
languages, such as those of propositional and predicate, a.k.a.,
first-order logic. Meta-logic investigates the properties of these
language, and of the canons of correct inference that use them. It
studies topics such as how to give precise meaning to the expressions
of these formal languages, how to justify the canons of valid
inference, what the properties of various proof systems are, including
their computational properties. These questions are important and
interesting in their own right, because the languages and proof
systems investigated are applied in many different areas---in
mathematics, philosophy, computer science, and linguistics,
especially---but they also serve as examples of how to study formal
systems in general. The logical languages we study here are not the
only ones people are interested in. For instance, linguists and
philosophers are interested in languages that are much more
complicated than those of propositional and first-order logic, and
computer scientists are interested in other \emph{kinds} of languages
altogether, such as programming languages. And the methods we discuss
here---how to give semantics for formal languages, how to prove
results about formal languages, how to investigate the properties of
formal languages---are applicable in those cases as well.

Like any discipline, metalogic both has a set of results or facts,
and a store of methods and techniques, and this text covers both.
Some students won't need to know some of the results we discuss
outside of this course, but they will need and use the methods we use
to establish them. The L\"owenheim-Skolem theorem, say, does not
often make an appearance in computer science, but the methods we use
to prove it do. On the other hand, many of the results we discuss do
have relevance for certain debates, say, in the philosophy of science
and in metaphysics. Philosophy students may not need to be able to
prove these results outside this course, but they do need to
understand what the results are---and you really only
\emph{understand} these results if you have thought through the
definitions and proofs needed to establish them. These are, in part,
the reasons for why the results and the methods covered in this text
are recommended study---in some cases even required---for students of
computer science and philosophy.

The material is divided into three parts. Part~1 concerns
itself with the theory of sets. Logic and metalogic is historically
connected very closely to what's called the ``foundations of
mathematics.''  Mathematical foundations deal with how ultimately
mathematical objects such as integers, rational, and real numbers,
functions, spaces, etc., should be understood. Set theory provides
one answer (there are others), and so set theory and logic have long
been studied side-by-side. Sets, relations, and functions are also
ubiquitous in any sort of formal investigation, not just in
mathematics but also in computer science and in some of the more
technical corners of philosophy. Certainly for the purposes of
formulating and proving results about the semantics and proof theory
of logic and the foundation of computability it is essential to have a
language in which to do this. For instance, we will talk about sets
of expressions, relations of consequence and provability,
interpretations of predicate symbols (which turn out to be relations),
computable functions, and various relations between and constructions
using these. It will be good to have shorthand symbols for
these, and think through the general properties of sets, relations,
and functions in order to do that. If you are not used to thinking
mathematically and to formulating mathematical proofs, then think of
the first part on set theory as a training ground: all the basic
definitions will be given, and we'll give increasingly complicated
proofs using them. Note that understanding these proofs---and being
able to find and formulate them yourself---is perhaps more important
than understanding the results, and especially in the first part, and
especially if you are new to mathematical thinking, it is important
that you think through the examples and problems.

In the first part we will establish one important result, however.
This result---Cantor's theorem---relies on one of the most striking
examples of conceptual analysis to be found anywhere in the sciences,
namely, Cantor's analysis of infinity. Infinity has puzzled
mathematicians and philosophers alike for centuries. No-one knew how
to properly think about it. Many people even thought it was a mistake
to think about it at all, that the notion of an infinite object or
infinite collection itself was incoherent. Cantor made infinity into
a subject we can coherently work with, and developed an entire theory
of infinite collections---and infinite numbers with which we can
measure the sizes of infinite collections---and showed that there are
different levels of infinity. This theory of ``transfinite'' numbers
is beautiful and intricate, and we won't get very far into it; but we
will be able to show that there are different levels of infinity,
specifically, that there are ``countable'' and ``uncountable'' levels
of infinity. This result has important applications, but it is
also really the kind of result that any self-respecting mathematician,
computer scientist, or philosopher should know.

In the second part we turn to first-order logic. We will define the
language of first-order logic and its semantics, i.e., what
first-order structures are and when a sentence of first-order logic is
true in a structure. This will enable us to do two important things:
(1)`We can define, with mathematical precision, when a sentence is a
logical consequence of another. (2)~We can also consider how the
relations that make up a first-order structure are
described---characterized---by the sentences that are true in them.
This in particular leads us to a discussion of the axiomatic method,
in which sentences of first-order languages are used to characterize
certain kinds of structures. Proof theory will occupy us next, and we
will consider the original version of the sequent calculus and natural
deduction as defined in the 1930s by Gerhard Gentzen. (Your instructor
may choose to cover only one, then any reference to ``derivations''
and ``provability'' will mean whatever system they chose.) The
semantic notion of consequence and the syntactic notion of provability
give us two completely different ways to make precise the idea that a
sentence may follow from some others. The soundness and completeness
theorems link these two characterization. In particular, we will prove
G\"odel's completeness theorem, which states that whenever a sentence
is a semantic consequence of some others, there it is also provable
from them. An equivalent formulation is: if a collection of sentences
is consistent---in the sense that nothing contradictory can be proved
from them---then there is a structure that makes all of them true.

The second formulation of the completeness theorem is perhaps the more
surprising. Around the time G\"odel proved this result (in 1929), the
German mathematician David Hilbert famously held the view that
consistency (i.e., freedom from contradiction) is all that mathematical
existence requires. In other words, whenever a mathematician can
coherently describe a structure or class of structures, then they
should be be entitled to believe in the existence of such structures.
At the time, many found this idea preposterous: just because you can
describe a structure without contradicting yourself, it surely does
not follow that such a structure actually exists. But that is exactly
what G\"odel's completeness theorem says. In addition to this
paradoxical---and certainly philosophically intriguing---aspect, the
completeness theorem also has two important applications which allow
us to prove further results about the existence of structures which
make given sentences true. These are the compactness and the
L\"owenheim-Skolem theorems.

In the third part \dots

% \section*{Acknowledgments}

% The material in the OLP used in
% \cref{inc:int::chap,cmp:rec::chap,inc:art::chap,inc:req::chap,inc:inp::chap}
% was based originally on Jeremy Avigad's lecture notes on
% ``Computability and Incompleteness,'' which he contributed to the OLP.
% I have heavily revised and expanded this material. The lecture notes,
% e.g., based theories of arithmetic on an axiomatic proof system. Here,
% we use Gentzen's standard natural deduction system (described in
% \cref{fol:nd:chap}), which requires dealing with trees primitive
% recursively (in \cref{cmp:rec:tre:sec}) and a more complicated
% approach to the arithmetization of !!{derivation}s (in
% \cref{inc:art:pnd:sec}).

% The biographies of logicians in \cref{bios:chap} and much of the
% material in \cref{fol:nd:chap} are originally due to Samara Burns.
% Dana H\"agg originally worked on the material in \cref{fol:part}.
