Hilbert's program aimed to show that all of mathematics could be
formalized in an axiomatized theory in a formal language, such as the
language of arithmetic or of set theory.  He believed that such a
theory would be \textbf{!!{complete}}. That is, for every !!{sentence}~$!A$,
either $\Th{T} \Proves !A$ or $\Th{T} \Proves \lnot !A$. In this sense
then, $\Th{T}$ would have settled every mathematical question: it
would either prove that it's true or that it's false. If Hilbert had
been right, it would also have turned out that mathematics is
\textbf{!!{decidable}}. That's because any !!{axiomatizable} theory
is \textbf{!!{computably enumerable}}, i.e., there is a computable
function that lists all its theorems. We can test if a
!!{sentence}~$!A$ is a theorem by listing all of them until we
find~$!A$ (in which it is a theorem) or $\lnot !A$ (in which case it
isn't). Alas, Hilbert was wrong. G\"odel proved that no
!!{axiomatizable}, consistent theory that is ``strong enough'' is
!!{complete}. That's the \textbf{first incompleteness theorem}. The
requirement that the theory be ``strong enough'' amounts to it
representing all computable functions and relations.  Specifically,
the very weak theory $\Th{Q}$ satisfies this property, and any theory
that is at least as strong as~$\Th{Q}$ also does. He also
showed---that is the \textbf{second incompleteness theorem}---that the
sentence that expresses the consistency of the theory is itself
undecidable in it, i.e., the theory proves neither it nor its
negation. So Hilbert's further aim of finding ``finitary'' consistency
proof of all of mathematics cannot be realized. For any finitary
consistency proof would, presumably, be formalizable in a theory that
captures all of mathematics. Finally, we established that theories
that represent all computable functions and relations are
not \textbf{!!{decidable}}. Note that although axomatizability and
completeness implies decidability, incompleteness does not imply
undecidability.  So this result shows that the second of Hilbert's
goals, namely that there be a procedure that decides if
$\Th{T} \Proves !A$ or not, can also not be achieved, at least not for
theories at least as strong as~$\Th{Q}$.
