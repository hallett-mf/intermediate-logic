In order to show that $\Th{Q}$ represents all computable functions, we
need a precise model of computability that we can take as the basis
for a proof.  There are, of course, many models of computability, such
as Turing machines. One model that plays a significant role
historically---it's one of the first models proposed, and is also the
one used by G\"odel himself---is that of the \textbf{recursive
  functions}.  The recursive functions are a class of arithmetical
functions---that is, their domain and range are the natural
numbers---that can be defined from a few basic functions using a few
operations. The basic functions are $\Zero$, $\Succ$, and the
projection functions. The operations are \textbf{composition},
\textbf{primitive recursion}, and \textbf{regular
  minimization}. Composition is simply a general version of ``chaining
together'' functions: first apply one, then apply the other to the
result. Primitive recursion defines a new function~$f$ from two
functions $g$, $h$ already defined, by stipulating that the value of
$f$ for $0$ is given by~$g$, and the value for any number $n+1$ is
given by $h$ applied to $f(n)$.  Functions that can be defined using
just these two principles are called \textbf{primitive recursive}. A
relation is primitive recursive iff its characteristic function is. It
turns out that a whole list of interesting functions and relations are
primitive recursive (such as addition, multiplication, exponentiation,
divisibility), and that we can define new primitive recursive
functions and relations from old ones using principles such as bounded
quantification and bounded minimization. In particular, this allowed
us to show that we can deal with \emph{sequences} of numbers in
primitive recursive ways. That is, there is a way to ``code''
sequences of numbers as single numbers in such a way that we can
compute the $i$-the element, the length, the concatenation of two
sequences, etc., all using primitive recursive functions operating on
these codes.  To obtain all the computable functions, we finally added
definition by \textbf{regular minimization} to composition and
primitive recursion.  A function~$g(x, y)$ is \textbf{regular} iff,
for every~$y$ it takes the value~$0$ for at least one~$x$. If $f$ is
regular, the least $x$ such that $g(x, y) = 0$ always exists, and can
be found simply by computing all the values of $g(0, y)$, $g(1, y)$,
etc., until one of them is $= 0$. The resulting function~$f(y) =
\umin{x}{g(x, y) = 0}$ is the function defined by regular minimization
from $g$. It is always total and computable. The resulting set of
functions are called \textbf{general recursive}.  One version of the
Church-Turing Thesis says that the computable arithmetical functions
are exactly the general recursive ones.
