In order to show how theories like~$\Th{Q}$ can ``talk'' about
computable functions---and especially about provability (via G\"odel
numbers)---we established that $\Th{Q}$ \textbf{represents} all
computable functions. By ``$\Th{Q}$ represents $f(n)$'' we mean that
there is !!a{formula}~$!A_f(x, y)$ in~$\Lang{L_A}$ which expresses
that $f(x) = y$, and $\Th{Q}$ can prove that it does.  This, in turn,
means that whenever $f(n) = m$, then $\Th{T} \Proves !A_f(\num{n},
\num{m})$ and $\Th{T} \Proves \lforall[y][(!A_f(\num{n}, y) \lif y =
\num{m})]$. (Here, $\num{n}$ is the \textbf{standard numeral} for~$n$,
i.e., the term $\Obj 0^{\prime\dots\prime}$ with $n$~$\prime$s. The
term~$\num{n}$ picks out the number~$n$ in the standard
model~$\Struct{N}$, so it's a convenient way of representing the
number~$n$ in $\Lang{L_A}$.) To prove that $\Th{Q}$ represents all
computable functions we go back to the characterization of computable
functions as those that can be defined from $\Zero$, $\Succ$, and the
projection functions, by composition, primitive recursion, and regular
minimization. While it is relatively easy to prove that the basic
functions are representable and that functions defined by composition
and regular minimization from representable functions are also
representable, primitive recursion is harder. We showed that we can
actually avoid definition by primitive recursion, if we allow a few
additional basic functions (namely, addition, multiplication, and the
characteristic function of~$=$). This required a \textbf{beta
function} which allows us to deal with sequences of numbers in a
rudimentary way, and which can be defined without using primitive
recursion.
