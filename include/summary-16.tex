A \textbf{model of arithmetic} is !!a{structure} for the
language~$\Lang{L_A}$ of arithmetic.  There is one distinguished such
model, the \textbf{standard model}~$\Struct{N}$, with $\Domain{N}
= \Nat$ and interpretations of $\Obj 0$, $\prime$, $+$, $\times$, and
$<$ given by $0$, the successor, addition, and multiplication
functions on~$\Nat$, and the less-than relation. $\Struct{N}$ is a
model of the theories $\Th{Q}$ and~$\Th{PA}$.

More generally, !!a{structure} for $\Lang{L_A}$ is
called \textbf{standard} iff it is isomorphic to~$\Struct{N}$. Two
!!{structure}s are isomorphic if there is an \textbf{isomorphism}
between them, i.e., !!a{bijective} function which preserves the
interpretations of !!{constant}s, !!{function}s, and !!{predicate}s.
By the \textbf{isomorphism theorem}, isomorphic structures
are \textbf{elementarily equivalent}, i.e., they make the same
sentences true.  In standard models, the domain is just the set of
values of all the numerals~$\num{n}$.

Models of $\Th{Q}$ and $\Th{PA}$ that are not isomorphic
to~$\Struct{N}$ are called \textbf{non-standard}. In non-standard
models, the domain is not exhausted by the values of the numerals.  An
element $x \in \Domain{M}$ where $x \neq \Value{\num{n}}{M}$ for
all~$n \in \Nat$ is called a \textbf{non-standard element}
of~$\Struct{M}$.  If $\Sat{M}{\Th{Q}}$, non-standard elements must
obey the axioms of~$\Th{Q}$, e.g., they have unique successors, they
can be added and multiplied, and compared using~$<$. The standard
elements of~$\Struct{M}$ are all $\Assign{<}{M}$ all the non-standard
elements.  Non-standard models exist because of the compactness
theorem, and for $\Th{Q}$ they can relatively easily be given
explicitly. Such models can be used to show that, e.g., $\Th{Q}$ is
not strong enough to prove certain !!{sentence}s, e.g.,
$\Th{Q} \Proves/ \lforall[x][\lforall[y][\eq[(x+y)][(y+x)]]]$. This is
done by defining a non-standard $\Struct{M}$ in which non-standard
elements don't obey the law of commutativity.

Non-standard models of~$\Struct{PA}$ cannot be so easily specified
explicitly. By showing that $\Th{PA}$ proves certain sentences, we can
investigate the structure of the non-standard part of a non-standard
model of~$\Th{PA}$. If a non-standard model~$\Struct{M}$ of~$\Th{PA}$
is !!{enumerable}, every non-standard element is part of a ``block''
of non-standard elements which are ordered like~$\Int$
by~$\Assign{<}{M}$.  These blocks themselves are arranged like~$\Rat$,
i.e., there is no smallest or largest block, and there is always a
block in between any two blocks.

Any !!{enumerable} model is isomorphic to one with domain~$\Nat$. If
the interpretations of $\prime$, $+$, $\times$, and $<$ in such a
model are computable functions, we say it is a \textbf{computable
model}. The standard model~$\Struct{N}$ is computable, since the
successor, addition, and multiplication functions and the less-than
relation on~$\Nat$ are computable. It is possible to define computable
non-standard models of~$\Th{Q}$, but $\Struct{N}$ is the only
computable model of~$\Th{PA}$. This is \textbf{Tannenbaum's Theorem}.
