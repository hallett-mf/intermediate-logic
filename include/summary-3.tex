A \emph{function}~$f\colon A \to B$ maps every element of the
\emph{domain}~$A$ to a unique element of the \emph{codomain}~$B$. If
$x \in A$, we call the $y$ that~$f$ maps $x$ to the
\emph{value}~$f(x)$ of $f$~for \emph{argument}~$x$. If $A$ is a set of
pairs, we can think of the function~$f$ as taking two arguments. The
\emph{range}~$\ran{f}$ of~$f$ is the subset of~$B$ that consists of
all the values of~$f$.

If $\ran{f} = B$ then $f$ is called \emph{surjective}. The value~$f(x)$
is unique in that $f$ maps $x$ to only one $f(x)$, never more than one.
If $f(x)$ is also unique in the sense that no two different arguments
are mapped to the same value, $f$ is called \emph{injective}. Functions
which are both injective and surjective are called \emph{bijective}.

Bijective functions have a unique \emph{inverse function}~$f^{-1}$.
Functions can also be chained together: the function $(g \circ f)$ is
the \emph{composition} of $f$ with $g$. Compositions of injective
functions are injective, and of surjective functions are surjective,
and $(f^{-1} \circ f)$ is the identity function.

If we relax the requirement that $f$ must have a value for every $x
\in A$, we get the notion of a \emph{partial functions}. If $f\colon
A \pto B$ is partial, we say $f(x)$ is \emph{defined},
$f(x) \fdefined$ if $f$ has a value for argument~$x$, and otherwise we
say that $f(x)$ is \emph{undefined}, $f(x) \fundefined$. Any (partial)
function~$f$ is associated with the \emph{graph}~$R_f$ of $f$, the
relation that holds iff $f(x) = y$.
