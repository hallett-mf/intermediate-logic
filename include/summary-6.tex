A \emph{first-order language} consists of \emph{constant},
\emph{function}, and \emph{predicate} symbols. Function and constant
symbols take a specified number of arguments. In the \emph{language of
arithmetic}, e.g., we have a single constant symbol~$\Obj 0$, one
1-place function symbol $\prime$, two 2-place function symbols~$+$ and
$\times$, and one 2-place predicate symbol~$<$. From \emph{variables}
and constant and function symbols we form the \emph{terms} of a
language. From the terms of a language together with its predicate
symbols, as well as the \emph{identity symbol}~$\eq$, we form the
\emph{atomic formulas}. And in turn from them, using the logical
connectives $\lnot$, $\lor$, $\land$, $\lif$, $\liff$ and the
quantifiers $\lforall$ and $\lexists$ we form its formulas. Since we
are careful to always include necessary parentheses in the process of
forming terms and formulas, there is always exactly one way of reading
a formula. This makes it possible to define things by induction on the
structure of formulas.

Occurrences of variables in formulas are sometimes governed by a
corresponding quantifier: if a variable occurs in the \emph{scope} of
a quantifier it is considered \emph{bound}, otherwise
\emph{free}. These concepts all have inductive definitions, and we
also inductively define the operation of \emph{substitution} of a term
for a variable in a formula. Formulas without free variable
occurrences are called \emph{sentences}.
