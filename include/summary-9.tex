\emph{Proof systems} provide purely syntactic methods for
characterizing consequence and compatibility between
sentences. \emph{Natural deduction} is one such proof
system. A \emph{derivation} in it consists of a tree formulas. The
topmost formulas in a derivation are \emph{assumptions}.  All other
formulas, for the derivation to be correct, must be correctly
justified by one of a number of \emph{inference rules}. These come in
pairs; an introduction and an elimination rule for each connective and
quantifier. For instance, if a formula~$!A$ is justified by a
$\Elim{\lif}$ rule, the preceding formulas (the \emph{premises}) must
be $!B \lif !A$ and $!B$ (for some~$!B$). Some inference rules also
allow assumptions to be \emph{discharged}. For instance, if $!A \lif
!B$ is inferred from $!B$ using $\Intro{\lif}$, any occurrences
of~$!A$ as assumptions in the derivation leading to the premise~$!B$
may be discharged, and is given a label that is also recorded at the
inference.

If there is a derivation with end formula~$!A$ and all assumptions are
discharged, we say $!A$ is a \emph{theorem} and write~$\Proves !A$. If
all undischarged assumptions are in some set~$\Gamma$, we say $!A$ is
\emph{derivable from}~$\Gamma$ and write $\Gamma \Proves !A$. If
$\Gamma \Proves \lfalse$ we say $\Gamma$ is \emph{inconsistent}, otherwise
\emph{consistent}. These notions are interrelated, e.g., $\Gamma
\Proves !A$ iff $\Gamma \cup \{\lnot !A\}$ is inconsistent. They are
also related to the corresponding semantic notions, e.g., if $\Gamma
\Proves !A$ then $\Gamma \Entails !A$. This property of proof
systems---what can be derived from $\Gamma$ is guaranteed to be
entailed by~$\Gamma$---is called \emph{soundness}. The \emph{soundness
theorem} is proved by induction on the length of derivations, showing
that each individual inference preserves entailment of its conclusion
from open assumptions provided its premises are entailed by their undischarged
assumptions.
