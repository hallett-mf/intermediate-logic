
\section{Tarski's Theorem and L\"ob's Theorem}

Tarski's Theorem shows that there's a gap between the notions of
``\emph{sentence provable in the theory}~$\Th{PA}$'' (focusing on
$\Th{PA}$ as the best theory of arithmetic we have and that of
`sentence true in $N$, the standard structure of natural numbers'.
The former notion is \emph{definable} in $\mathcal{L}_{A}$, and the
latter is \emph{not definable}. Remember, `true in $N$' is really the
same as `provable in $\Th{TA}$', so the difference, if you want to
put it this way, is between these two notions of provable. 

Tarski's paper ``Truth and proof'' \citep{Tarski1969} makes the
following interesting remarks, which emphasize this gap, but which put
an optimistic gloss on it:

\begin{quotation}
Nothing is detracted from the significance of this result [these
results] by the fact that its philosophical implications are
essentially negative in character. \emph{The result shows indeed that
in no domain of mathematics is the notion of provability a perfect
substitute for the notion of truth. The belief that formal proof can
serve as an adequate instrument for establishing truth of all
mathematical statements has proved to be unfounded.} The original
triumph of formal methods has been followed by a serious setback.

Whatever can he said to conclude this discussion is bound to be an
anticlimax. The notion of truth for formalized the­ories can now be
introduced by means of a precise and ad­equate definition. It can
therefore be used without any re­strictions and reservations in
metalogical discussion. It has actually become a basic metalogical
notion involved in im­portant problems and results. On the other hand,
the notion of proof has not lost its significance either. \emph{Proof
is still the only method used to ascertain the truth of sentences
within any specific mathematical theory. We are now aware of the fact,
however, that there are sentences formulated in the language of the
theory which are true but not provable, and we cannot discount the
possibility that some such sentences occur among those in which we are
interested and which we attempt to prove.} Hence in some situations we
may wish to explore the possibility of widening the set of provable
sen­tences. To this end we enrich the given theory by including new
sentences in its axiom system or by providing it with new rules of
proof. In doing so we use the notion of truth as a guide; for we do
not wish to add a new axiom or a new rule of proof if we have reason
to believe that the new axiom is not a true sentence, or that the new
rule of proof when applied to true sentences may yield a false
sentence. The process of extending a theory may of course be repeated
arbitrarily many times. The notion of a true sentence functions thus
as an ideal limit which can never be reached but which we try to
approximate by gradually widening the set of provable sentences. (It
seems likely, although for different reasons, that the notion of truth
plays an analogous role in the realm of empirical knowledge.) There is
no conflict between the notions of truth and proof in the development
of mathematics; the two notions are not at war but live in peaceful
coexistence. \citep[p.~77, emphasis added]{Tarski1969}.
\end{quotation}
What Tarski brings out here is the gap between truth and proof, and
the fact that there must be, in effect, a dialectical relationship
between the two, perhaps more than what he calls here `peaceful
coexistence'. This gap was dramatically emphasised by G\"odel, in  a
lecture in 1951, in a comment he made about the Second Incompleteness
Theorem, a comment which supports Tarski's point that we should use
``truth as a guide'':
\begin{quote}
It is \emph{this} [the second] theorem which makes the
incompletability of mathematics particularly evident. For, \emph{it
makes it impossible that someone should set up a certain well-defined
system of axioms and rules and consistently make the following
assertion about it: All of these axioms and rules I perceive (with
mathematical certitude) to be correct, and moreover I believe that
they contain all of mathematics}. If someone makes such a statement he
contradicts himself.\footnote{[G\"odel's footnote:] If he only says
``I believe I shall be able to perceive one after the other to be
true'' (where their number is supposed to be infinite), he does not
contradict himself.} For if he perceives the axioms under
consideration to be correct, he also perceives (with the same
certainty) that they are consistent. Hence he has a mathematical
insight not derivable from his axioms. However, one has to be careful
in order to understand clearly the meaning of this state of affairs.
Does it mean that no well-defined system of correct axioms can contain
all of mathematics proper? It does, if by mathematics proper is
understood the system of all true mathematical propositions; it does
not, however, if one understands by it the system of all demonstrable
mathematical propositions. \citep[p.~309]{Godel1951}
\end{quote}

The juxtaposition of Tarski's Theorem and the G\"odel results show us
that the gap between truth and provability is large; what L\"ob's
Theorem does is emphasise how large that gap really is, and in
particular how unlike the notion of truth the notion of proof is.
Above all it shows that  proof can't really act as a surrogate for
truth, that provability can't be just a form of ``supertruth'', like
logical truth or necessary truth.  

Let us show how L\"ob's Theorem brings out the difference. 

First,  if we proceeded with truth in much the same way as we proceed
in the prove of L\"ob's Theorem we would get something quite absurd, a
proof of the existence of Santa Claus. Note the fact that we have
``Santa Claus exists'' here plays no role whatsoever. It could be
``The Dark Lord exists'', or any one of your favourite nonsense
claims, e.g., ``The Moon is made of green cheese''. We presented a
version of this argument in the textbook. Here is the same argument
given by George Boolos in his book \emph{The Logic of Provability}:
\begin{quote} 
Let Sam be the sentence ``if Sam is true, SC'' [where ``SC''
abbreviates the sentence ``Santa Claus exists'']. Assume that Sam is
true; then ``if Sam is true, SC'' is true; thus if Sam is true, SC;
and so SC by modus ponens. Thus we have shown that SC on the
assumption that Sam is true and have therefore shown outright that if
Sam is true, SC. But then ``If Sam is true, SC'' is true, i.e., Sam is
true, and by modus ponens again, SC. \citep[p.~56]{Boolos1993}
\end{quote}
To reiterate,  \emph{any} statement can be proved this way. What this
means is that we can easily get a contradiction, which means that we
must be operating with inconsistent assumptions! But which
inconsistent assumptions? In fact, the argument above is just an
elaborate version of the Liar Paradox, and we are in fact operating
with the same inconsistent assumption as is behind that.

We can show this as follows. If you look carefully at the argument
Boolos presents, you will notice that several times we make use of the
principle:
\begin{equation*}\label{eqn:convention-t}
\text{``$X$'' is true iff $X$}\tag{T}
\end{equation*}
which we saw earlier illustrated through the particular instance
``\,`Snow is white' is true iff snow is white''. We use the principle
here in the shift (left-to-right) from ``{}`If Sam is true, SC' is
true'' to ``If Sam is true, SC''. Then later we have the shift
(right-to-left here) from ``If Sam is true, SC'' to ``{}`If Sam is
true, SC' is true''. The principle (T) (often referred to as
``convention~(T)'') is the  basic principle that a correct
truth-definition has to satisfy. Moreover, it is the principle which
in English (taken as the meta-language for discussion of English)
allows us to deduce a contradiction from ``This sentence is false'',
i.e., the statement which gives us the Liar Paradox. 

Let's now go back to Tarski's Theorem in the form that we proved it,
i.e., showing that no $1$-place predicate of the language can define
the set of G\"odel-numbers of sentences true in the standard model.
Assume that there were such a predicate~$!D(x)$. We would have that
$n$ is the G\"odel number of sentence $!A$ (so $n= \Gn{!A}$) true in
the standard model if, and only if, $\Sat{N}{!D(\gn{!A})}$. From this
follows $\Sat{N}{!A}$ iff $\Sat{N}{!D(\gn{!A})}$. But this means that
$\Sat{N}{!A \liff !D(\gn{!A})}$. And  remember that saying
``$\Sat{N}{!X}$'' is the same as saying ``$\Th{TA} \Proves !X$'. Put
all this together and we have:
\begin{equation*}\label{eqn:t-for-ta}
 \Th{TA} \Proves !A \liff !D(\gn{!A}),\tag{T$_{\Th{TA}}$}
\end{equation*}
which, if we read $!D(x)$ as ``is true'', really says that $\Th{TA}$
can produce the principle \cref{eqn:convention-t} above. We often say
that principles like  \cref{eqn:t-for-ta} are ``truth-definitions'',
holding as they do for all sentences of the language concerned. But
this immediately gives rise here to a contradiction, for we can apply
the Diagonal Lemma to $\lnot !D(x)$ and get a sentence~$!L$ such that
\begin{align*}
    \Th{TA} &\Proves !L \liff \lnot !D(\gn{!L}).
\intertext{But \cref{eqn:t-for-ta} above instantiated for~$!L$
will give us:}
\Th{TA} & \Proves !L \liff !D(\gn{!L}),
\end{align*}
thus contradiction. 

Let's come back to the difference between truth and provability.
First, as we've seen, what the principle \cref{eqn:convention-t}
allows is switching  as we please between ``{}'$X$' is true''
and~``$X$''. But the reasoning presented in the proof of L\"ob's
Theorem ``mimics'' the Santa Claus argument when ``is true'' replaced
by ``is provable'', and where the requisite switching between
``{}'$X$' is provable'' and ``$X$'' is now provided by the
derivability conditions. We do not, of course, get a contradiction. 

Second, we can use L\"ob's Theorem to give us Tarski's Theorem in the
form that we cannot have a truth-definition, i.e., that ``true in
$\Struct N$'' cannot be definable.  The argument goes roughly like
this:

Suppose $\mathrm{Tr}(x)$ is a truth-predicate for an appropriate
theory $\Th{T}$, and the truth-definition  
\begin{equation}\label{eqn:loeb-td}
\Th{T}\Proves A \liff \mathrm{Tr}(\gn{A})\tag{\textit{Tr}}
\end{equation}
holds. Then it's clear using this that all of the derivability
conditions hold for $\mathrm{Tr}(x)$, which means that we must have a
L\"ob Theorem for $\mathrm{Tr}(x)$. (Remember, in the statement of
L\"ob's Theorem, all we really need to know about the provability
predicate is that the derivability conditions hold for it. We know
nothing else about the ``inner workings'' of the predicate. Thus, if
there were a truth-predicate, it would satisfy the conditions of
L\"ob's Theorem.) Now take \textit{any} sentence~$!A$. Then by
\eqref{eqn:loeb-td} we have $\Th{T} \Proves \mathrm{Tr}(\gn{!A}) \liff
!A$, so in particular $\Th{T} \Proves \mathrm{Tr}(\gn{!A}) \lif !A$.
By L\"ob's Theorem applied to $\mathrm{Tr}(x)$, this means that we
have $\Th{T} \Proves !A$. So \emph{any} sentence would be provable in
$\Th{T}$ if $\Th{T}$ were equipped with a truth-definition, which of
course would make $\Th{T}$ inconsistent. So a sufficient condition for
a sentence~$!A$ to be provable is that $\Th{T} \Proves
\mathrm{Tr}(\gn{A}) \lif A$, but \eqref{eqn:loeb-td} says that that
must be the case for any sentence.
%\footnote{Part of what we're saying here is that, as far as L\"ob's
%Theorem goes, the only thing that's really important about
%`$\OProv(x)$' is that it satisfies the three Hilbert-Bernays
%derivability conditions , which amounts to saying that we can prove a
%L\"ob Theorem for any 1-place predicate which satisfies analogous
%conditions.} 
Hence,  using L\"ob's Theorem, we've shown that if $\Th{T}$ is
consistent, there can't be a truth-definition for~$\Th{T}$. 

In short, if we could get L\"ob for truth, we would get contradiction,
but L\"ob for provability \emph{does not} give us contradiction. But
if go further into the comparison, we can see  how wide the gap is. 

Boolos has some  comments on the remarkable nature of the L\"ob
Theorem which touch on what we've just observed, and which emphasize
just how big the gap really is. In reading these comments, first think
of ``$\OProv(x)$'' as a natural replacement for ``truth$(x)$''; that will
highlight the surprise. Boolos comments as follows:
\begin{quotation}
L\"ob's theorem is utterly astonishing for at least five reasons. In
the first place, it is often hard to understand how vast the
mathematical gap is between truth and provability. And to one who
lacks that understanding and does not distinguish between truth and
provability, $\OProv(\gn{S})\lif S$, which the hypothesis of
L\"ob's theorem asserts to be provable, might appear to be trivially
true in all cases, whether $S$ is true or false, provable or
unprovable. But if $S$ is false,  $S$ had better not be provable. Thus
it would seem that $S$ ought not always to be provable provided merely
that (the possibly trivial-seeming) $\OProv(\gn{S})\lif S$ is
provable.

Secondly, $\OProv$ seems here to be working like negation. After
all, if $\lnot S \lif S$ is provable, then so is $S$; proving $S$ by
proving $\lnot S\lif S$ is called \emph{reductio ad absurdum} (or,
sometimes, the law of Clavius). Moreover, inferring $S$ solely on the
ground that ($S\lif S$) is demonstrable is known as begging the
question, or reasoning in a circle. To one who conflates truth and
provability, it may then seem that L\"ob's theorem asserts that
begging the question is an admissible form of reasoning in PA.

Thirdly, one might have thought that \emph{at least on occasion}, PA
would claim to be sound with regard to an unprovable sentence $S$,
i.e., claim that \emph{if} it proves $S$, then $S$ holds. But L\"ob's
theorem tells us that it never does so: PA makes the claim
$\OProv(\gn{S})\lif S$ that it is sound with regard to $S$ only
when it obviously must, when the consequent $S$ is actually provable.
As Rohit Parikh once put it, ``PA couldn't be more modest about its
own veracity''.

Fourthly, one might very naturally suppose that provability is a kind
of necessity, and therefore, just as $\Box(\Box p \lif p)$ always
expresses a truth if the box is interpreted as ``it is necessary
that'' ---~for then $\Box(\Box p \lif p)$ says that it is necessarily
true that if a statement is necessarily true, it is true
---~$\OProv(\gn{\OProv(\gn{S})\lif S})$ would also always
be true or at least true in some cases in which $S$ is false and not
true only in the rather exceptional cases in which $S$ is actually
provable.

Finally, it seems wholly bizarre that the statement that if $S$ is
provable, then $S$ is true is not itself provable, in general. For
isn't it perfectly obvious, for any $S$, that $S$ is true if provable?
Why are we bothering with PA if its theorems are false? And how could
any such (apparently) obvious truth not be provable?
\citep[pp.~54--55]{Boolos1993}\footnote{We've replaced Boolos' use
of G\"odel's \textit{Bew} with our `$\OProv$'.}
\end{quotation}

%Here, we can fill out some of these points with the following highly
%relevant observation. 
%
%With the informal argument (for SC), we were implicitly using a
%truth-definition, for we were constantly going back and forth between
%`$\varphi$' and `\,``$\varphi$'' is true' (so in effect between
%$T(\gn{\varphi})$ and $\varphi$).  And in fact, this is what we do in
%the argument for the various Liar Paradoxes. But notice that L\"ob's
%Theorem does not use a truth predicate like this, but rather a
%provability predicate, `$\OProv(x)$'. Indeed, we can use
%L\"ob's Theorem \emph{to show} that there \emph{cannot} be a
%truth-definition for $S$ if $S$ is consistent. 
%
%The argument goes roughly like this:
%
%Suppose $\mathrm{Tr}(x)$ is a truth-predicate for $S$, i.e.,  the
%truth-definition  
%
%\begin{equation}\label{loeb-td} S\Proves \mathrm{Tr}(\gn{\varphi})
%\liff \varphi \end{equation} holds. Then it's clear using this that
%all of the derivability conditions hold for $\mathrm{Tr}(x)$, which
%means that we must have a L\"ob Theorem for $\mathrm{Tr}(x)$. Now
%take \textit{any} sentence $\varphi$. Then by (\ref{loeb-td}) we have
%$S\Proves \mathrm{Tr}(\gn{\varphi}) \liff \varphi$, so $S\Proves
%\mathrm{Tr}(\gn{\varphi}) \lif \varphi$. By L\"ob's Theorem, this
%means that we have $\Proves \varphi$. So \emph{any} sentence would be
%provable in $S$ if $S$ were equipped with a truth-definition, which
%of course would make $S$ inconsistent.\footnote{Part of what we're
%saying here is that, as far as L\"ob's Theorem goes, the only thing
%that's really important about `$\OProv(x)$' is that it
%satisfies the three Hilbert-Bernays provability condition, which
%amounts to saying that we can prove a L\"ob Theorem for any 1-place
%predicate which satisfies analogous conditions.} 
%
%Hence,  using L\"ob's Theorem, we've shown that  if $S$ is
%consistent, there can't be a truth-definition for  $S$. 


Let's elaborate a little on some of these points.

The first point. Take an example like ``$2+2=5$''. We certainly can't
have it that $\Th{PA}$ can prove $2+2=5$, which means (following
L\"ob's Theorem) that it can't be the case that $\Th{PA} \Proves
\OProv(\gn{2+2=5}) \to 2+2=5$!{} In other words, $\Th{PA}$ can't
tell us that whatever it proves is true, or that the notion of proof
it recognizes is a good one!{} And haven't we designed it so that the
formal provability predicate ``$\OProv(x)$'' (for $\Th{PA}$)
matches ``is provable in~$\Th{PA}$''?

The third point extends this. Suppose we take something like the
Goldbach Conjecture (\textit{GC\/}) or the Twin Prime Conjecture
(\textit{TPC\/}), propositions that we don't know to be provable
in~$\Th{PA}$. We would hope that $\Th{PA}$ would be able to prove of
such propositions, at least some of the time, that
$\OProv(\gn{\textit{GC\/}}) \to \textit{GC}$ or
$\OProv(\gn{\textit{TPC\/}}) \to \textit{TPC}$, i.e., that it knows a
$\Th{PA}$ proof of $\textit{TPC}$ would show that $\textit{TPC}$ is
correct. But L\"ob's Theorem tells us it can't do that, or rather that
it can only do that if there is \emph{already} a $\Th{PA}$ proof of
\textit{GC} or~\textit{TPC}!{}

Think about how different this is from truth. For a truth-definition
(if we have one), it has got to be the case that $\mathrm{Tr}(\gn{A})
\liff A$ holds regardless of whether $A$ is true, false, contradictory
or ridiculous: and therefore $\mathrm{Tr}(\gn{A}) \lif A$ holds  also,
regardless of what $A$ asserts. In other words (switching to English)
``{}`The moon is made of green cheese' is true $\Leftrightarrow$ [or
even just $\Rightarrow$]  the moon is made of green cheese'', even
though the sentence involved here is not true, and is even faintly
ludicrous, one we're only prepared to contemplate by reason of it's
syntactic formulation.  But L\"ob's Theorem tells us that ``$\Th{T}
\Proves S$'' cannot be a surrogate for truth in this sense, for we
could then, it seems,  prove that the moon is made of green cheese (or
any other kind of cheese, or marshmallow or sphagnum moss \dots), or
some arithmetic equivalents. That shows us (Tarski's Theorem) that we
can't have a truth-definition for $\Th{PA}$ where (T$_{\Th{PA}}$) is
provable in $\Th{PA}$. But if ``provable'' is a surrogate for ``is
true'', then we we would certainly expect $\Th{PA}$ to be able to
prove ``$\OProv(\gn{A}) \lif A$'' in general. This is stressed in
Point 5.

Lastly, consider the second point. As Boolos points out, suppose we
think that ``$\OProv(\gn{S})$'' is really some sort of strong
affirmation of $S$ (``being true and more''). In this case, then
``$\OProv(\gn{S}) \lif S$'' would be  something like ``$S \lif S$''
(or at least this should follow from ``$\OProv(\gn{S}) \lif S$'').
But then using this as a \emph{justification} for ``$\Th{T}\Proves S$''
is really like arguing by begging the question!{} However,  if
``$\OProv(\gn{S}) \lif S$'' really acts as a genuine justification
for $S$,  then it looks as if ``$\OProv(\gn{S})$'' is behaving,
not like an affirmation at all, but really rather like like a
\emph{negation}, as in the proof of~``$S$'' given by proving ``$\lnot S
\lif S$''!{} So it seems as if it can't be the case that
``$\OProv(\gn{S})$'' is like a strong \emph{affirmation} of $S$
(``being true and more'') at all, and is actually more like a
\emph{denial} of~$S$.
